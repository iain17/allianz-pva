\chapter{Projectgrenzen}
%Opschrijven wat je niet doet in je project geeft helderheid over wat je wel gaat doen. Hierdoor kun je ook voorkomen dat stakeholders tijdens het project met eisen komen die echt buiten de opdracht vallen. In deze paragraaf baken je je project dus af. In elk geval ga je hierbij in op hoe lang het project duurt, dus wanneer is het project klaar. Wat hoort inhoudelijk niet meer tot de opdracht? Bijvoorbeeld: ontwikkelen voor Windows, maar niet voor Linux; of wel een concept uitdenken, maar niet realiseren.
De afstudeeropdracht gaat van start op 5 februari 2018 en zal eindigen op 29 juni 2018. Het is de bedoeling dat ik gedurende de stageperiode aan de afstudeeropdracht en verslagen voor het afstuderen werk. Tijdens het onderzoek en analyse wordt enkel aandacht besteed aan onderwerpen die van belang zijn en een duidelijke toegevoegde waarde leveren aan het onderzoek en de proof of concept applicatie. Om de analyse en onderzoek correct en op tijd af te kunnen ronden wordt er aan alle onnodige zaken en informatie geen aandacht besteed.\par
Het project betreft alleen de use-case van Allianz en de mogelijkheden om het via de blockchain technologie op te lossen. Andere technologieën zijn uitgesloten, waarbij er wel binnen blockchain naar verschillende blockchain oplossingsrichtingen wordt gekeken. Er wordt tijdens dit afstudeerproject uiteindelijk maar één proof of concept opgeleverd en gepresenteerd aan de opdrachtgever. Enige nazorg na de stageperiode, of het hosten van software is niet van toepassing op dit project.

\section{Minimum Viable Product}
De Minimum Viable Product (MVP) voor het POC is als volgt voor dit project:
\begin{itemize}
  \item Broker: opstellen van co-insurance contracten (smart contract \footnote{https://en.wikipedia.org/wiki/Smart\_contract})
  \item Verzekerde: claim aanvragen
  \item Verzekeraar: claims goed of afkeuren
  \item Verzekeraar: automatische goed of afkeuren op basis van business ruling
\end{itemize}