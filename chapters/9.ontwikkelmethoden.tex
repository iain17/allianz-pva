\chapter{Ontwikkelmethoden}
%Nu je weet wat je gaat opleveren (producten en kwaliteit daarvan) en wat je daarvoor moet doen (overzicht activiteiten) met welke grenzen en randvoorwaarden, kun je bedenken wat de beste methode is om alles te realiseren. Heb je te maken met een ingevuld adviestraject dan ligt het voor de hand om eerst een onderzoek te doen en daarna met een advies te komen.
%Ben je bezig met het ontwerp van bijvoorbeeld een website of ga je een stuk software ontwikkelen, maak dan eerst een onderbouwde keuze tussen bijvoorbeeld waterval of incrementeel/iteratief. Daarbij spelen in elk geval de volgende overwegingen een rol:
%- In hoeverre kunnen de resultaten van het project snel en volledig worden beschreven?
%- Welke methoden hanteert het bedrijf en in hoeverre kun of moet je daar bij aansluiten? 
%Welke methoden ken je vanuit de opleiding?
%Wees hierbij wél kritisch: tijdens het afstuderen werk je bijvoorbeeld meestal in je eentje, en niet iedere methode (bijvoorbeeld Scrum bij het ontwikkelen van software) leent zich om individueel mee aan de slag te gaan. Soms is het handig om een methode daarop aan te passen. Dat kan, als je het maar goed onderbouwt en je daarbij baseert op betrouwbare bronnen. En als de methode is voorgeschreven: onderbouw waarom jij vindt dat deze methode passend is bij het soort project dat je moet gaan uitvoeren. Onderaan dit document vind je enkele suggesties voor literatuur over ontwikkelmethoden.
Hoofd en deelvragen, onderzoeksmethode. Projectmangement methode