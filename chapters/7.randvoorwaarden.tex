\chapter{Randvoorwaarden}
%Succesvol zijn gaat maar zelden ‘zomaar’. Bij randvoorwaarden geef je aan welke zaken geregeld moeten zijn zodat je zelf vlot door kunt werken. Denk bijvoorbeeld aan:
%- Ritme van de opleiding dicteert (denk hierbij bijvoorbeeld aan inleverdeadlines)
%- De opdracht gaat voor
%- Beschikbaarheid begeleider bedrijf. Is hij bijvoorbeeld op de juiste momenten beschikbaar
%voor het geven van feedback en het tijdig nemen van beslissingen?
%- Welke resources (b.v. soft- of hardware) moeten er zijn om te kunnen werken? Denk aan
%toegang tot systemen, ontwikkelsoftware, aanschaffen van benodigde licenties
%- Et cetera
Aan het afstudeerproject zijn enkele randvoorwaarden verbonden om een succesvol verloop te bevorderen. Deze randvoorwaarden zijn belangrijk voor de afbakening van het project. Hieronder staan de randvoorwaarden gegroepeerd weergegeven.\par

\section{m.b.t. afstudeerbedrijf}
\begin{itemize}
  \item Het project moet starten op 5 februari 2018;
  \item Het project moet op 29 juni 2018 afgerond zijn;
  \item De presentatie van het afstudeerproject moet gepland worden tussen 14 mei en 1 juni 2018;
  \item Het afstudeerproject wordt projectmatig en methodisch uitgevoerd.
  \item De stagiair besteedt 40 uur per week aan dit project;
  \item De uit te voeren activiteiten bestaan in principe alleen uit de afstudeer activiteiten die beschreven staan in het plan van aanpak;
  \item Het afstudeerproject wordt zelfstandig uitgevoerd en toont bekwaamheid competenties aan vanuit de courses in de studie;
  \item De stagair moet de bedrijfsbegeleider kunnen aanspreken in geval van onduidelijkheden;
  \item De bedrijfsbegeleider is aanwezig bij de bespreking van het concept van het projectplan met de docent begeleider;
  \item Aan het eind van het afstudeerproject vult de bedrijfsbegeleider een beoordelingsformulier over het functioneren van de afstudeerder in;
  \item De bedrijfsbegeleider is aanwezig is bij het mondelinge tentamen / de presentatie en verdediging van het afstudeerproject op school.
\end{itemize}

\newpage
\section{m.b.t. school}
\begin{itemize}
% \item Na elke twee weken wordt de voortgang gerapporteerd aan de school begeleider.
\item Geeft feedback op het projectplan (plan van aanpak, onderzoeksplan);
\item Geeft feedback op de 80\% versie (reflectieverslag, onderzoeksverslag);
\item De schoolbegeleider bezoekt het stagebedrijf waarbij een gesprek plaatsvindt met de stagiair en de stagebegeleider;
\item Na het einde van de stageperiode vindt de afstudeerpresentatie plaats op school.
\end{itemize}