\chapter{Op te leveren producten en kwaliteitseisen}
%In dit hoofdstuk behandel je alle producten die je in hoofdstuk 5 beschreven hebt en moet opleveren, zoveel mogelijk in detail. Het gaat dan zowel om producten die je aan je opdrachtgever levert, als om de producten die de opleiding van je vraagt. Daarin verdeel je de resultaten die je moet opleveren een in kleinere (deel)producten. Zo kan het resultaat ‘een stuk werkende code’ bestaan uit een ontwerp, code, een testrapport en overdrachtsdocumentatie, etc.
De producten die ten behoeve van het afstudeeropdracht worden opgeleverd, dienen van voldoende kwaliteit te zijn. Om dit te garanderen zijn de volgende kwaliteitseisen beschreven in de onderstaande tabel.

\small
\begin{center}
 \begin{tabular}{|c c c c|} 
 
 \hline
 Product & Kwaliteitseisen & Activiteiten & Proceskwaliteit \\ [0.5ex] 
 \hline
 Plan van Aanpak & \makecell{
 - Voldoet aan ICA\\Controlekaart \cite{icaControl}\\
 - Voldoet aan ho-\\ofdstuk beschrijvingen\\ICA\cite{pvaTut}
 } & \makecell{
 - Schrijven van PVA\\Hoofdstukken.\\
 - Projectgrenzen definiëren. \\
 - Plannen.\\
 - Risco's in kaart brengen.
 } & \makecell{
 - Draft laten reviewen\\door minstens twee\\deskundigen\\
 - Een draft versie\\opsturen in week 2.
 } \\
 \hline

 \hline
 \makecell{
 Onderzoeksverslag\\
 } & \makecell{
 - Voldoet aan ICA\\Controlekaart \cite{icaControl}\\
 - Leidt af van Theo's\\ handleiding \cite{theoOnderzoek}
 } & \makecell{
 - Literatuuronderzoek \\
 - Interviews, experimenten \\
 - Schrijven hoofdstukken \cite{icaOnderzoek}:\\
 Inleiding,\\
 Resultaat hoofdstukken,\\
 Conclussie,\\
 Literatuurlijst
 } & \makecell{
 - Draft laten reviewen\\door minstens twee\\deskundigen\\
 - Draft versies opsturen naar\\ stagebegeleider
 } \\
 \hline
 
 \hline
 Eindpresentatie & \makecell{
 - Voldoet aan ICA\\Controlekaart \cite{icaControl}\\
 - Vat het project\\samen
 } & \makecell{
 Geven van presentatie\\
 Maken presentatie:\\
 Opdrachtomschrijving,\\
 Process en resultaten,\\
 Conclussie.
 } & \makecell{
 - Draft laten reviewen\\door minstens twee\\deskundigen
 } \\
 \hline
 
 \hline
 Reflectieverslag & \makecell{
 - Voldoet aan ICA\\Controlekaart \cite{icaControl}
 } & \makecell{
 Schrijven hoofdstukken:\\
 Inleiding,\\
 Opdrachtomschrijving\\
 Methode,\\
 Process en resultaten,\\
 Conclussie, Discussie\\
 Reflectie\\
 }& \makecell{
 - Draft laten reviewen\\door minstens twee\\deskundigen
 } \\
 \hline
 
 \hline
 Code & \makecell{
% -Unittests\\
 -Commentaar in het\\ Engels\\
 -Gebruik versiebeheer
 } & \makecell{
 -Schrijven code\\
 - SOLID}
% -Unit tests schrijven
 } & \makecell{
 - Schrijven van user stories\\
 - Statische code analyse\\
 - Refactoring
 } \\
 \hline
\end{tabular}
\end{center}
\end{small}

\newpage
\section{Competenties}
%De volgende stap is dat je kijkt of je met de producten die je moet opleveren je competenties voldoende kunt aantonen. Is dat niet het geval dan voeg je aan de lijst met ‘de producten voor de opleiding’ die onderdelen toe die nodig zijn om je opdracht ook voor de opleiding voldoende inhoud te geven. In elk geval staan de volgende zaken ook op die lijst: afstudeerverslag (inclusief reflectie en bijlagen), eindpresentatie, en voor semesterstudenten het onderzoeksrapport inclusief onderzoeksplan (voor profielers niet verplicht).
Om aan het einde en tijdens het afstudeerproject aan te kunnen tonen dat ik op aspirant hbo-niveau het project heb uitgevoerd. Behandeld dit hoofdstuk de producten die opgeleverd worden en welke competenties hiermee worden aangetoond. Deze competenties komen uit de OER studie ICA handleiding 2017-2018 \cite{studiegids} en zijn in bijlagen \ref{chap:competenties} terug te vinden.

\begin{itemize}
  \item \textbf{SD-1: Software Requirements} User stories, tickets, Proof of Concept, Analyse naar huidige bedrijfsprocessen in onderzoek
  \item \textbf{SD-2: Software Design} Onderzoeksverslag, Reflectieverslag
  \item \textbf{SD-3: Software Architecture} Onderzoeksverslag
  \item \textbf{SD-4: Software Construction} Code, Proof of Concept
  \item \textbf{SD-5: Software Testing and Quality} Code, Proof of Concept
  \item \textbf{SD-6: Software Engineering Process and Management}  Plan van Aanpak, Onderzoeksverslag
  \item \textbf{SD-7: Research} Onderzoeksverslag, reflectieverslag
  \item \textbf{SD-8: Self support} Plan van aanpak, Onderzoeksverslag, Eindpresentatie
\end{itemize}

%Door de onderzoek naar requirements van de huidige bedrijfsprocessen worden functionele en niet functionele requirements verzameld. Door het maken van user stories in de Trello tickets (cards) worden verdere requirements verzameld.