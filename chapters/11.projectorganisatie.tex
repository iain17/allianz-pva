\chapter{Projectorganisatie en communicatie}
%Nu de verlangde resultaten en ontwikkelmethode bekend zijn kun je pas de projectorganisatie ( in 9.) en de planning (in 10.) behandelen. Dit hoofdstuk dat inzicht geeft in contactfrequenties tussen jou, de organisatie en de afstudeerdocent. Ga in elk geval in op:
%- Wie zijn je begeleiders (de opleiding en bedrijf)
%- Hoe vaak heb je contact met hen en waarover?
%- Wie is waarvoor verantwoordelijk
%- Wat zijn ieders –inclusief je eigen- contactgegevens?
De afstudeeropdracht wordt uitgevoerd door de stagiair waarbij ondersteuning wordt gegeven door de bedrijfsbegeleider. Daarnaast vindt er terugkoppeling plaats met de stagebegeleider die gezamenlijk samen met de assessor het project beoordeeld. De verdere projectorganisatie en context is beschreven in hoofdstuk \ref{chap:context}. De contactgegevens van de betrokkenen zijn hieronder weergegeven:\par
\paragraph{Stagair}~\\
Naam:		Calum Iain Munro\\
Telefoon:	06 835 426 80\\
E-mail:		CI.Munro@student.han.nl

\paragraph{Bedrijfsbegeleider \& Product owner}~\\
Naam:		Dani\"el Siahaya\\
Telefoon:	06 421 060 92\\
E-mail: 	Daniel@HeadFWD.com

\paragraph{Klant}~\\
Naam:		Arjan Zaal\\
Telefoon:	-\\
E-mail: 	arjan.zaal@allianz.nl

\paragraph{Stagebegeleider}~\\
Naam:		Misja Nabben\\
Telefoon:	06 553 720 69\\
E-mail:		Misja.Nabben@Han.nl

\paragraph{Assessor}~\\
Naam:		Rein Harle\\
Telefoon:	-\\
E-mail: 	Leon.bronckers@Han.nl
