\chapter{Risico's}
%Dit hoofdstuk is een soort ‘final check’. De zaken die je kunt voorkomen door wijzigingen aan te brengen in de planning neem je alsnog op in je planning. Denk bijvoorbeeld aan voldoende overlegmomenten met je opdrachtgever. Alléén de risico’s die je niet vooraf kunt beïnvloeden neem je op in deze paragraaf. Een voorbeeld: als je weet dat je tijdens je project gaat verhuizen kun je in je planning opnemen op welke dagen je niet werkt. Afwezigheid door verhuizing is dus geen risico. Maar, als je afhankelijk bent van de levering van een server door een nieuwe leverancier, kan het anders zijn. Natuurlijk neem je eerst in de planning op dat je er nog een keer extra achter aan belt (dit noemen we een tegenmaatregel), maar je vraagt je ook af wat je gaat doen als het onverhoopt tóch misgaat (je uitwijkstrategie).
%Dit kun je weergeven in een tabel met als koppen: risico, kans (groot-middel-klein), impact (groot- middel-klein), tegenmaatregel, uitwijkstrategie.
In dit hoofdstuk worden de risico's tijdens het afstudeerproject benoemd en daar waar mogelijk opgevangen door een oplossing.\par

\section{Interne risico's}
Interne risico's vallen binnen de scope van het project of de verantwoordelijkheid van de projectorganisatie. In deze paragraaf worden de interne risico's benoemd.\par
\begin{itemize}
\item (Langdurige) ziekte van de stagiair. Dit wordt tijdig aangegeven aan de stagebegeleider en assessor, met verwachte datum van terugkomst. Daarnaast wordt dit risico gereduceerd door in de planning gebruik te maken van uitlooptijden.
\item Langdurige ziekte van de stagebegeleider. In dit geval dient direct een nieuwe assessor aangewezen te worden.
\item Langdurige ziekte van de bedrijfsbegeleider. In dit geval neemt een andere gekwalificeerde werknemer van HeadForward de begeleiding over. Dit risico wordt deels opgevangen door het feit dat er meerdere gekwalificeerde werknemers aanwezig zijn.
\item Een foutieve inschatting van de projectgrootte/projectduur. Het is mogelijk dat het project meer of minder tijd kost dan de uiteindelijke duur van het afstudeerproject. Mocht het project langer blijken te duren dan de totale afstudeertijd dan dient de opdracht in nauw overleg met de stagebegeleider en bedrijfsbegeleider dusdanig aangepast te worden dat deze in de beschikbare tijd kan worden afgerond. Mocht de opdracht voor het einde van het afstudeerproject voltooid zijn, dan verricht de stagiair in nauw overleg met de stagebegeleider en bedrijfsbegeleider aanvullende werkzaamheden.
\item Een gebrek aan begeleiding. De mogelijkheid bestaat dat de stagebegeleider en/of bedrijfsbegeleider onvoldoende tijd beschikbaar heeft voor het voldoende begeleiden van de stagiair. Dit kan afgevangen worden door begeleiding te zoeken bij andere werknemers.
\end{itemize}

\newpage
\section{Externe risico's}
Externe risico's liggen buiten de scope van het project en de projectorganisatie. In deze paragraaf worden de externe risico's benoemd.\par
\begin{itemize}
\item Onvolwassen technieken. De blockchain wordt momenteel nog amper tot niet gebruik in productieomgevingen. De documentatie en ondersteuning van sommige functies kan gebrekkig zijn waardoor het POC lastig te realiseren is. Hiervoor wordt er dan gezamenlijk met externe opdrachtgever gekeken naar alternatieven.
\item Verlies van gegevens door brand, diefstal en/of falende apparatuur. Door gebruik te maken van een versiebeheersysteem waarbij de software op meer dan één locatie opgeslagen is, wordt dit risico grotendeels afgedekt. Mochten er desondanks toch gegevens verloren gaan dan wordt hiervoor, in nauw overleg met de stagebegeleider, een oplossing gezocht.
\item Langdurige ziekte van de opdrachtgever. In dit geval neemt een andere gekwalificeerde werknemer van Allianz deze rol over. Dit risico wordt deels opgevangen door het feit dat er meerdere gekwalificeerde werknemers aanwezig zijn.
\end{itemize}
