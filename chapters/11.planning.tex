%\begin{sidewaysfigure}

\chapter{Planning}
%In dit hoofdstuk maak je een koppeling tussen de ontwikkelmethode en je activiteiten. Dit geef je weer in een GANTT-chart oftewel strokenplanning, waarin je je mijlpalen duidelijk weergeeft. Let op dat je ontwikkelmethode voldoende herkenbaar is in de planning.
\begin{ganttchart}[vgrid, hgrid]{1}{21}
\gantttitle{februari}{4}
\gantttitle{maart}{4}
\gantttitle{april}{4}
\gantttitle{mei}{5}
\gantttitle{juni}{4}\\
\gantttitlelist{6,...,26}{1}\\
%Projectplan
\ganttgroup{Project plan}{1}{6} \\
\ganttbar{Plan Van Aanpak}{1}{4} \\
\ganttbar{Onderzoeksplan}{2}{4} \\
\ganttmilestone{Concept projectplan}{3}\\
\ganttmilestone{Definitief projectplan}{6}\\
%Onderzoek
\ganttgroup{Onderzoek}{5}{18} \\
\ganttbar{Analysis}{10}{16} \\
\ganttbar{Requirements}{10}{16} \\
\ganttbar{Prototyping}{10}{16} \\
%Afstudeerverslagen
\ganttgroup{Afstudeerverslagen}{10}{20} \\
\ganttbar{Reflectieverslag}{10}{19} \\
\ganttbar{Presentatie}{18}{19} \\
\ganttmilestone{80\% versie}{15}\\
\ganttmilestone{Eindverslag}{19}\\
\ganttmilestone{Afstudeerpresentatie}{20}\\

\node [left] at ([yshift=-1.3cm]) {Week};
\end{ganttchart}
%\end{sidewaysfigure}

\newpage
\section{Opleverdata mijlpaalproducten}
In deze paragraaf zijn de uiterlijke opleverdata voor de mijlpaalproducten vermeld.

\small
\begin{center}
 \begin{tabular}{|c c|} 
 
 \hline
 Product & Datum \\ [0.5ex] 
 \hline
 Projectplan concept (Plan van Aanpak, Onderzoeksplan) & 23-02-2018 \\
 \hline
 Projectplan definitief (Plan van Aanpak, Onderzoeksplan) & 16-03-2018 \\
 \hline
 80\% versie (Onderzoeksverslag, POC, Reflectieverslag) & 14-05-2018 \\
 \hline
 Eindverslag (Onderzoeksverslag, POC, Reflectieverslag) & 11-06-2018 \\
 \hline
\end{tabular}
\end{center}
\end{small}
