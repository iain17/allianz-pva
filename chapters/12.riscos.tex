\chapter{Risico's}
%Dit hoofdstuk is een soort ‘final check’. De zaken die je kunt voorkomen door wijzigingen aan te brengen in de planning neem je alsnog op in je planning. Denk bijvoorbeeld aan voldoende overlegmomenten met je opdrachtgever. Alléén de risico’s die je niet vooraf kunt beïnvloeden neem je op in deze paragraaf. Een voorbeeld: als je weet dat je tijdens je project gaat verhuizen kun je in je planning opnemen op welke dagen je niet werkt. Afwezigheid door verhuizing is dus geen risico. Maar, als je afhankelijk bent van de levering van een server door een nieuwe leverancier, kan het anders zijn. Natuurlijk neem je eerst in de planning op dat je er nog een keer extra achter aan belt (dit noemen we een tegenmaatregel), maar je vraagt je ook af wat je gaat doen als het onverhoopt tóch misgaat (je uitwijkstrategie).
%Dit kun je weergeven in een tabel met als koppen: risico, kans (groot-middel-klein), impact (groot- middel-klein), tegenmaatregel, uitwijkstrategie.
In dit hoofdstuk worden de risico's tijdens het afstudeerproject benoemd en daar waar mogelijk opgevangen door een oplossing.\par

\section{Interne risico's}
Interne risico's vallen binnen de scope van het project of de verantwoordelijkheid van de projectorganisatie. In deze paragraaf worden de interne risico's benoemd.\par
\begin{itemize}
\item (Langdurige) ziekte van de stagiair. Dit wordt tijdig aangegeven aan de stagebegeleider en assessor, met verwachte datum van terugkomst. Daarnaast wordt dit risico gereduceerd door in de planning gebruik te maken van uitlooptijden.
\item Langdurige ziekte van de stagebegeleider. In dit geval dient direct een nieuwe assessor aangewezen te worden.
\item Langdurige ziekte van de bedrijfsbegeleider. In dit geval neemt een andere gekwalificeerde werknemer van HeadForward de begeleiding over. Dit risico wordt deels opgevangen door het feit dat er meerdere gekwalificeerde werknemers aanwezig zijn.
\item Een foutieve inschatting van de projectgrootte/projectduur. Het is mogelijk dat het project meer of minder tijd kost dan de uiteindelijke duur van het afstudeerproject. Mocht het project langer blijken te duren dan de totale afstudeertijd dan dient de opdracht in nauw overleg met de stagebegeleider en bedrijfsbegeleider dusdanig aangepast te worden dat deze in de beschikbare tijd kan worden afgerond. Mocht de opdracht voor het einde van het afstudeerproject voltooid zijn, dan verricht de stagiair in nauw overleg met de stagebegeleider en bedrijfsbegeleider aanvullende werkzaamheden.
\item Een gebrek aan begeleiding. De mogelijkheid bestaat dat de stagebegeleider en/of bedrijfsbegeleider onvoldoende tijd beschikbaar heeft voor het voldoende begeleiden van de stagiair. Dit kan afgevangen worden door begeleiding te zoeken bij andere werknemers.
%\item Opleveren van incorrecte testsoftware. Door diverse oorzaken is het mogelijk dat er een fout in de testsoftware terecht komt. Hierdoor kunnen foutieve testresultaten ontstaan die zich vervolgens kunnen uiten in schade aan apparatuur. Dit risico is voor een groot deel af te vangen door de testsoftware te laten evalueren door andere medewerkers/stagiairs.
%\item Wijzigingen in de software waarop de testsoftware gebaseerd is. De bestaande software is onderhevig aan wijzigingen, hierdoor ontstaat het risico dat de testsoftware onvolledig wordt of dat er fouten ontstaan. Dit risico wordt opgevangen door wijzigingen in de software direct in kaart te brengen en door te voeren in de testscenario's en de testsoftware. Daarnaast wordt gedocumenteerd hoe de testsoftware onderhouden dient te worden waardoor wijzigingen in de software die plaatsvinden na het afstudeerproject eveneens doorgevoerd kunnen worden in de testsoftware.
%\item De logistieke laag bestaat uit diverse threads waardoor het risico bestaat dat de testsoftware moeilijk te realiseren is en inconsistente resultaten geeft. Wanneer deze situatie zich voordoet dient de stagiair zich verder te verdiepen in JUnittesten waarbij gebruik wordt gemaakt van threads en met name in methodes die worden toegepast bij multithreaded JUnittesten.
%\item De testscenario's kunnen bestaan uit scenario's die in werkelijkheid over een langere periode getest dienen te worden. Door deze langere periode ontstaat het risico dat de testsoftware er te lang over doet om binnen de gestelde tijd op de betreffende scenario's te testen. Dit risico is op te vangen door de tijd in de testsoftware te versnellen waardoor alle functionaliteiten alsnog binnen de gestelde tijd getest kunnen worden.
%\item Het risico bestaat dat de testsoftware voor de SpectroManager niet vertaald kan worden naar XML. In dit geval wordt er gezocht naar alternatieve oplossingen. De mogelijke oplossingen worden vervolgens met de stagebegeleider besproken om het beste alternatief te bepalen en om te controleren of deze oplossing haalbaar is binnen de planning.
\end{itemize}

\section{Externe risico's}
Externe risico's liggen buiten de scope van het project en de projectorganisatie. In deze paragraaf worden de externe risico's benoemd.\par
\begin{itemize}
%\item Falende apparatuur. Er bestaat een mogelijkheid dat de apparatuur niet naar behoren werkt. In dit geval wordt de systeembeheerder direct op de hoogte gesteld zodat deze voor vervangende apparatuur kan zorgen.
\item Verlies van gegevens door brand, diefstal en/of falende apparatuur. Door gebruik te maken van een versiebeheersysteem waarbij de software op meer dan één locatie opgeslagen is, wordt dit risico grotendeels afgedekt. Mochten er desondanks toch gegevens verloren gaan dan wordt hiervoor, in nauw overleg met de stagebegeleider, een oplossing gezocht.
%\item Het verloren gaan van belangrijke kennis van het systeem door het vertrek van een medewerker. Dit kan worden opgevangen door meer tijd te investeren in de analyse van de programmatuur. Hierdoor ontstaat echter het risico dat het analyseren van de software meer tijd kost dan gepland. Om dit op te vangen zal in nauw overleg met de stagebegeleider bepaald worden hoe dit zal worden opgevangen.
\end{itemize}
