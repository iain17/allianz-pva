\chapter{Risico's}
Dit hoofdstuk is een soort ‘final check’. De zaken die je kunt voorkomen door wijzigingen aan te brengen in de planning neem je alsnog op in je planning. Denk bijvoorbeeld aan voldoende overlegmomenten met je opdrachtgever. Alléén de risico’s die je niet vooraf kunt beïnvloeden neem je op in deze paragraaf. Een voorbeeld: als je weet dat je tijdens je project gaat verhuizen kun je in je planning opnemen op welke dagen je niet werkt. Afwezigheid door verhuizing is dus geen risico. Maar, als je afhankelijk bent van de levering van een server door een nieuwe leverancier, kan het anders zijn. Natuurlijk neem je eerst in de planning op dat je er nog een keer extra achter aan belt (dit noemen we een tegenmaatregel), maar je vraagt je ook af wat je gaat doen als het onverhoopt tóch misgaat (je uitwijkstrategie).
Dit kun je weergeven in een tabel met als koppen: risico, kans (groot-middel-klein), impact (groot- middel-klein), tegenmaatregel, uitwijkstrategie.