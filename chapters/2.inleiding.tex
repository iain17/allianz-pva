\chapter{Inleiding}
%Introductie bedrijf/organisatie, korte introductie van de opdracht, leeswijzer: wat volgt in de rest van het document.
Dit document betreft het plan van aanpak van het afstudeerproject "Allianz - Automatiseren Claim Process", die ik Iain Munro ga uitvoeren tijdens 2017/2018 periode 3. De Afstudeeropdracht wordt uitgevoerd in opdracht van Allianz via het stagebedrijf HeadForward. Dit document geeft invulling aan het project en beschrijft globaal hoe dit ten uitvoer wordt gebracht.

Dit plan van aanpak is in de eerste plaatst bedoeld voor de opdrachtgever en anderen geïnteresseerden om inzicht te krijgen in het project, zodat de vooruitgang van het project kan worden bewaakt. Daarnaast biedt het inzicht voor de school begeleider, assessor als indicatie voor de kwaliteit van de afstudeeropdracht.

Dit document is opgedeeld in 12 hoofdstukken. Na deze inleiding volgt de achtergrond informatie van het stagebedrijf en de afstudeeropdracht. Vervolgens wordt de aanleiding beschreven en komt de daaruit voortvloeiende afstudeeropdracht aan bod. Aan de hand van de afstudeeropdracht worden de doelstellingen geformuleerd, gevolgd door een concrete beschrijving van de activiteiten die hieruit voortkomen.

In het hierop volgende hoofdstuk worden de methoden en technieken die gebruikt worden tijdens het project gedefinieerd. Vervolgens worden de uiteindelijk op te leveren producten als resultaat van de project activiteiten beschreven. Daarnaast is het hoofdstuk "Project grenzen" toegevoegd, waarin een aantal afspraken zijn opgenomen om duidelijkheid te geven over wat wel en wat niet binnen het afstudeerproject valt.

Voor het afstuderen op HBO niveau wordt in het hoofdstuk "Op te leveren producten en kwaliteitseisen" de competenties behandeld die ik aantoon tijdens dit project, om hiermee aan te kunnen tonen HBO waardig te zijn. Het hoofdstuk "Planning" bevat een globale planning voor het afstudeerproject en geeft inzicht in het afstudeertraject. Verder zijn er enkele afspraken opgenomen voor het waarborgen van de kwaliteit en wordt inzicht gegeven in de organisatie die betrokken is bij het project.

In het hoofdstuk "Randvoorwaarden" wordt aangegeven wat de kostenposten van het project zijn en wat het project uiteindelijk oplevert. In het laatste hoofdstuk zijn eventuele risico's opgenomen in combinatie met de manier waarop deze risico 's opgevangen kunnen worden.