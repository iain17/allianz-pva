\chapter{Op te leveren producten en kwaliteitseisen}
%In dit hoofdstuk behandel je alle producten die je in hoofdstuk 5 beschreven hebt en moet opleveren, zoveel mogelijk in detail. Het gaat dan zowel om producten die je aan je opdrachtgever levert, als om de producten die de opleiding van je vraagt. Daarin verdeel je de resultaten die je moet opleveren een in kleinere (deel)producten. Zo kan het resultaat ‘een stuk werkende code’ bestaan uit een ontwerp, code, een testrapport en overdrachtsdocumentatie, etc.
De producten die ten behoeve van het afstudeeropdracht worden opgeleverd, dienen van voldoende kwaliteit te zijn. Om dit te garanderen zijn de volgende kwaliteitseisen beschreven in de onderstaande tabel.

\small
\begin{center}
 \begin{tabular}{|c c c c|} 
 
 \hline
 Product & Kwaliteitseisen & Activiteiten & Proceskwaliteit \\ [0.5ex] 
 \hline
 Plan van Aanpak & \makecell{
 - Voldoet aan ICA\\Controlekaart \cite{icaControl}\\
 - Voldoet aan ho-\\ofdstuk beschrijvingen\\ICA\cite{pvaTut}
 } & \cite{pvaTut} & \makecell{
 - Draft laten reviewen\\door minstens twee\\deskundigen
 } \\
 \hline

 \hline
 \makecell{
 Onderzoeksverslag\\
 } & \makecell{
 - Voldoet aan ICA\\Controlekaart \cite{icaControl}\\
 - Leidt af van Theo's\\ handleiding \cite{theoOnderzoek}
 } & \makecell{\cite{theoOnderzoek}\\\cite{icaOnderzoek}} & \makecell{
 - Draft laten reviewen\\door minstens twee\\deskundigen
 } \\
 \hline
 
 \hline
 Eindpresentatie & \makecell{
 - Voldoet aan ICA\\Controlekaart \cite{icaControl}
 } & \makecell{Opdrachtomschrijving,\\Process en resultaten,\\Conclussie} & \makecell{
 - Draft laten reviewen\\door minstens twee\\deskundigen
 } \\
 \hline
 
 \hline
 Reflectieverslag & \makecell{
 - Voldoet aan ICA\\Controlekaart \cite{icaControl}
 } & \makecell{
 Hoofdstukken: Inleiding,\\
 Opdrachtomschrijving\\
 Methode,\\
 Process en resultaten,\\
 Conclussie, Discussie\\
 Reflectie\\
 }& \makecell{
 - Draft laten reviewen\\door minstens twee\\deskundigen
 } \\
 \hline
 
 \hline
 Code & \makecell{
 -Unittests\\
 -Commentaar in het\\ Engels\\
 -Gebruik versiebeheer
 } & \makecell{
 -Schrijven code\\
 -Unit tests schrijven
 } & \makecell{
 - Statische code analyse\\
 - Refactoring
 } \\
 \hline
\end{tabular}
\end{center}
\end{small}

\newpage
\section{Competenties}
%De volgende stap is dat je kijkt of je met de producten die je moet opleveren je competenties voldoende kunt aantonen. Is dat niet het geval dan voeg je aan de lijst met ‘de producten voor de opleiding’ die onderdelen toe die nodig zijn om je opdracht ook voor de opleiding voldoende inhoud te geven. In elk geval staan de volgende zaken ook op die lijst: afstudeerverslag (inclusief reflectie en bijlagen), eindpresentatie, en voor semesterstudenten het onderzoeksrapport inclusief onderzoeksplan (voor profielers niet verplicht).
Om aan het einde en tijdens het afstudeerproject aan te kunnen tonen dat ik op aspirant hbo-niveau het project heb uitgevoerd. Behandeld dit hoofdstuk de competenties die ik tijdens het afstuderen ga aantonen. Deze competenties zijn een directe kopie uit de Eindkwalificaties van de OER studie ICA handleiding 2017-2018 \cite{studiegids} en worden gekoppeld aan de eindproducten van dit project.

\subsection{SD-1: Software Requirements}
\textbf{Producten:} Proof of Concept \\
De student analyseert en specificeert de eisen aan een ICT-oplossing op basis van de gebruikersbehoeften op een gestructureerde en gestandaardiseerde manier. Valideert de opgestelde eisen.
Beheert (veranderende) eisen tijdens het software- ontwikkeltraject.

\subsection{SD-2: Software Design}
\textbf{Producten:} Onderzoeksverslag \\
De student kan op basis van de requirements de interne structuur – de elementen en hun relaties - van een data- intensief en gedistribueerd softwaresysteem bepalen, zowel op top-level niveau (architectuur) als ook op gedetailleerd niveau (ontwerp).
\\
De student kan de gemaakte ontwerpkeuzes onderbouwen, past tijdens het ontwerpen standaard notaties en best practices uit het beroepenveld toe, en houdt in het ontwerp rekening met mogelijke onderhoudsvragen.

\subsection{SD-3: Software Architecture}
\textbf{Producten:} Onderzoeksverslag \\
De student kan op basis van de non-functional requirements de interne structuur op top-level niveau van een data-intensief en gedistribueerd softwaresysteem bepalen.
\\
De student kan de gemaakte architecturele keuzes onderbouwen en past tijdens het ontwerpen van de architectuur best practices uit het beroepenveld toe.

\subsection{SD-4: Software Construction}
\textbf{Producten:} Code, Proof of Concept \\
De student kan op basis van een ontwerp werkende en betekenisvolle data- intensieve en gedistribueerde software systemen realiseren, schrijft begrijpbare en hoogwaardige source code en past professionele tools en technieken toe om dit te bereiken, en kan in teamverband een volledig geïntegreerd en systeem opleveren, dat klaar is voor ingebruikname

\subsection{SD-5: Software Testing and Quality}
\textbf{Producten:} Code, Proof of Concept \\
De student kan aantonen dat het systeem aan de geïdentificeerde requirements voldoet en dat de opgeleverde producten, onder andere de source code, aan vooraf gedefinieerde kwaliteitscriteria voldoen.

\newpage
\subsection{SD-6: Software Engineering Process and Management}
\textbf{Producten: Plan van Aanpak, Onderzoeksverslag, Onderzoeksplan} \\
De student kan in een multidisciplinaire omgeving op grond van de gekozen ontwikkelmethodiek, passend bij de context en inhoud van de opdracht, een software-ontwikkeltraject projectmatig inrichten en uitvoeren, kiest geschikte methoden en technieken, past deze toe, en bewaakt de voortgang van het project door gebruik te maken van procesondersteunende tools.

\subsection{SD-7: Research}
\textbf{Producten:} Onderzoeksverslag \\
De student kan een probleem op het terrein van Software Development (bijvoorbeeld inzet van nieuwe technologieën) oplossen doo reen kleinschalig onderzoek uit te voeren op een systematische, methodisch verantwoorde wijze, en kan de conclusies daaruit onderbouwen en effectief communiceren.

\subsection{SD-8: Self support}
\textbf{Producten:} Plan van aanpak, Onderzoeksverslag, Eindpresentatie \\
De student kan als een beginnende professional zelfstandig een authentieke beroepsopdracht uitvoeren die leidt tot een of meer beroepsproducten en de uitvoering ervan verantwoorden.
