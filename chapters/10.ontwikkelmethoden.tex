\chapter{Ontwikkelmethoden}\label{chap:projectmethod}
%Nu je weet wat je gaat opleveren (producten en kwaliteit daarvan) en wat je daarvoor moet doen (overzicht activiteiten) met welke grenzen en randvoorwaarden, kun je bedenken wat de beste methode is om alles te realiseren. Heb je te maken met een ingevuld adviestraject dan ligt het voor de hand om eerst een onderzoek te doen en daarna met een advies te komen.
%Ben je bezig met het ontwerp van bijvoorbeeld een website of ga je een stuk software ontwikkelen, maak dan eerst een onderbouwde keuze tussen bijvoorbeeld waterval of incrementeel/iteratief. Daarbij spelen in elk geval de volgende overwegingen een rol:
%- In hoeverre kunnen de resultaten van het project snel en volledig worden beschreven?
%- Welke methoden hanteert het bedrijf en in hoeverre kun of moet je daar bij aansluiten? 
%Welke methoden ken je vanuit de opleiding?
%Wees hierbij wél kritisch: tijdens het afstuderen werk je bijvoorbeeld meestal in je eentje, en niet iedere methode (bijvoorbeeld Scrum bij het ontwikkelen van software) leent zich om individueel mee aan de slag te gaan. Soms is het handig om een methode daarop aan te passen. Dat kan, als je het maar goed onderbouwt en je daarbij baseert op betrouwbare bronnen. En als de methode is voorgeschreven: onderbouw waarom jij vindt dat deze methode passend is bij het soort project dat je moet gaan uitvoeren. Onderaan dit document vind je enkele suggesties voor literatuur over ontwikkelmethoden.
Gedurende het project wordt de agile softwareontwikkelmethode Kanban gehanteerd. Kanban is een lean fabricagemethoden (geïnspireerd op het Toyota Production System) en wordt voornamelijk gebruikt voor softwareontwikkeling en technologie gerelateerd werk. Kanban kan echter op elk werkgebied worden toegepast en kan zelfs worden gecombineerd met andere methoden of ontwikkelmethodieken zoals Scrum.\footnote{https://en.wikipedia.org/wiki/Kanban\_(development)}\par

Door aantal projectkenmerken is er voor Kanban gekozen. Kanban beperkt zich echter niet tot alleen Software development en past dus ook gelijk bij het onderdeel onderzoek in het project. Verder is er voor Kanban gekozen omdat het project niet in teamverband wordt uitgevoerd. Kanban in tegenstelling tot Scrum vereist niet de teamvergaderingen die normaal gesproken nodig zijn als het project in teamverband werd uitgevoerd (daily standups, retrospectives etc..).\par

Daarbij heb ik ervaring met Kanban en biedt deze iteratieve aanpak de mogelijkheid om het werk op te splitsen, in te schatten en prioriteiten te stellen in iteraties van 2 weken. Dit geeft de vereiste flexibiliteit om snel te kunnen schakelen bij aanpassingen. Gezien er onderzoek wordt uitgevoerd en er gebruik gemaakt wordt van experimentele software kan er gelijk begonnen worden aan een stuk onderzoek waarbij de bevindingen gelijk getest kunnen worden in het prototype (proof of concept).\par

Gedurende het project wordt er ook iteratief verbeterd. Na elke iteratie wordt er samen met de bedrijfsbegeleider gekeken naar het proces en mogelijke verbeteringen. Om deze feedback loop zo kort mogelijk te houden is er gekozen voor 2 weken iteraties en een Kanban WIP\footnote{https://kanbanize.com/kanban-resources/getting-started/what-is-wip/} limiet van één. Dit omdat het project niet in teamverband wordt uitgevoerd.