\chapter{Inleiding}
%Introductie bedrijf/organisatie, korte introductie van de opdracht, leeswijzer: wat volgt in de rest van het document.
Dit document betreft het plan van aanpak van het afstudeerproject ``\thesisTitle'', die ik ga uitvoeren van 5 februari 2018 tot 29 juni 2018 (2017/2018 periode 3). De Afstudeeropdracht wordt uitgevoerd in opdracht van Allianz via het stagebedrijf HeadForward. De afstudeeropdracht is het ontwikkelen van een 'proof of concept' voor opdrachtgever Allianz, waarbij vooraf analyse en onderzoek wordt uitgevoerd.\pa r

Het plan van aanpak is in de eerste plaats bedoeld voor de opdrachtgever en andere geïnteresseerden om inzicht te krijgen in het project, zodat de vooruitgang van het project kan worden bewaakt. Daarnaast biedt het inzicht voor de schoolbegeleider, assessor als indicatie voor de kwaliteit van de afstudeeropdracht.\par

Dit document is opgedeeld in 12 hoofdstukken. Na deze inleiding volgt de achtergrondinformatie van het stagebedrijf en de afstudeeropdracht, die inzicht geeft in de organisaties die betrokken zijn bij het project. Vervolgens wordt de aanleiding beschreven en komt de daaruit voortvloeiende afstudeeropdracht aan bod. Aan de hand van de afstudeeropdracht worden de doelstellingen geformuleerd, gevolgd door een concrete beschrijving van de activiteiten die hieruit voortkomen.\par

In het volgende hoofdstuk worden de projectgrenzen gedefinieerd. In dit hoofdstuk worden een aantal afspraken opgenomen, om duidelijkheid te geven over wat wel en wat niet binnen het afstudeerproject valt. Vervolgens worden de methoden en technieken die gebruikt worden tijdens het project beschreven.\par

Daarnaast is het hoofdstuk ``Op te leveren producten en kwaliteitseisen''  toegevoegd, waarin alle products- en kwaliteitseisen zijn opgenomen voor het waarborgen van de kwaliteit. Voor het afstuderen op HBO niveau worden in dit hoofdstuk ook de competenties behandeld die ik aantoon tijdens het project, om hiermee mijn bekwaamdheid aan te kunnen tonen. \par

In een vervolg hoofdstuk wordt de ontwikkelmethoden behandeld die gehanteerd wordt tijdens het afstudeerproject. Het hoofdstuk ``Planning'' bevat een globale planning voor het afstudeerproject en geeft inzicht in het afstudeertraject.\par

In het laatste hoofdstuk zijn eventuele risico's opgenomen in combinatie met de manier waarop deze risico's opgevangen kunnen worden.